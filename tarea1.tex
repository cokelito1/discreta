\documentclass[a4paper,oneside,10.5pt]{USMArt}

\usepackage{personal}
\usepackage{comment}
\usepackage[letterpaper, left=2cm, right=2cm]{geometry}

\title{Tarea I - Matematica Discreta}
\sigla{MAT-379 }
\ramo{Matematica Discreta}
\profesor{Roberto Villaflor}
\semestre{2025-1}
\author{Jorge Bravo}

\begin{document}
\maketitle

\begin{sol}
  Dada una coloracion $C$ sobre $G$, consideramos
  \begin{equation*}
    f(C) = |\{ e \in E(G) \; | \; \text{ Los extremos de } e \text{ tienen el mismo color }\}|
  \end{equation*}

  Tomemos una coloración que minimiza esta función. Supongamos que $C$ no cumple la propiedad para algún vértice
  $v \in V(G)$, esto significa que m\'as de la mitad de los vértices adyacentes tienen el mismo color, cambiando
  el color de $v$ tenemos que la cantidad de aristas que conectan vértices del mismo color disminuyo, pues teníamos
  que m\'as de la mitad de los vecinos de $v$ tenían su color anterior. Por lo tanto una coloración que minimice
  $f$ a de cumplir la propiedad que queremos.
\end{sol}

\begin{sol}
  Considere $v$ un vértice con $d(v) = |N(v)|$. Luego hay una única arista por cada vecino, consideremos para cada
  arista $e$ incidente en $v$ un camino maximal $P_{v}$. Estos caminos han de terminar en una hoja, pues de
  otra forma lo podemos continuar, pues de otra forma se armaría un ciclo si llegaramos a un vértice que ya visitamos.
  Estos caminos no se pueden intersectar, pues si se intersectaran entonces en el primer vértice que se intersectan
  podríamos armar un ciclo con ambos caminos. Dado que tenemos $\Delta(G)$ caminos disjuntos que terminan en hojas,
  a lo menos tenemos $|\Delta(G)|$ hojas.
\end{sol}

\begin{sol}
  Notemos que resultado es trivial para $K_{3}$, pues $K_{3}$ es un triangulo por lo que 2 de las aristas tienen
  que tener el mismo color y por tanto esas 2 forman un árbol generador del mismo color. Consideramos $K_{n + 1}$, para
  $n > 3$, considerando el subgrafo conformado por $G - v_{n + 1}$, este es isomorfo a $K_{n}$ y cumple la propiedad
  por lo tanto existe un árbol generador $T$ del mismo color, digamos $C$, para este subgrafo. Si existiese una arista
  incidente en $v_{n + 1}$ de color $C$, entonces uniendo $v_{n + 1}$ al árbol mediante esa arista tendríamos el árbol
  generador. Si no existiese ninguna arista incidente en $v_{n + 1}$ de color $C$, entonces todas las aristas incidentes
  en $v_{n + 1}$ son del mismo color, esto pues dado $e, f$ aristas incidentes en $v_{n + 1}$ estas conectan
  $v_{n + 1}$ con $v_{e}$ y $v_{f}$ respectivamente, dado que estas están conectadas por el árbol $T$, existe un camino
  $v_{e}, v_{1}, \dots, v_{f}$, de color $C$, considerando los triángulos formados por vértices
  adyacentes en el camino y $v_{n + 1}$, tenemos que los $2$ vértices incidentes en $v_{n + 1}$ tienen que tener mismo
  color, pues ninguno puede tener color $C$. Repitiendo esto tenemos que todas las aristas que conectan $v_{n + 1}$
  con vértices del camino tienen que tener el mismo color, en particular $e$ y $f$. Por lo tanto en ese caso
  el árbol generador es el grafo generado por todos los vértices y las aristas incidente en $v_{n + 1}$.
\end{sol}

\end{document}
