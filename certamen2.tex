\documentclass[a4paper,oneside,10.5pt]{USMArt}

\usepackage{personal}
\usepackage{comment}
\usepackage[letterpaper, left=2cm, right=2cm]{geometry}

\title{Certamen II - Variedades Diferenciables}
\sigla{MAT-430 }
\ramo{Variedades Diferenciables}
\profesor{Pedro Montero}
\semestre{2025-1}
\author{Jorge Bravo}

\DeclareMathOperator{\volg}{vol_g}
\DeclareMathOperator{\dive}{div}
\begin{document}
\maketitle

\newcommand{\TT}{\mathbf{T}}

\begin{sol}\hfill
  \begin{enumerate}
  \item Notemos que
  \begin{align*}
    d(\iota_{X} \volg) &= d(\sqrt{\det(g_{ij})} \sum_{i = 1}^{n} (-1)^{i} X^{i} dx^{1} \wedge \dots \wedge \hat{dx^{i}} \wedge dx^{n})\\
                       &= d(\sum_{i = 1}^{n} (-1)^{i} \sqrt{\det (g_{\ell j})} X^{i} dx^{1} \wedge \dots \wedge \hat{dx^{i}} \wedge dx^{n})\\
     &= \sum_{i = 1}^{n} \frac{\partial}{\partial x^{i}} (\det(g_{\ell j}) X^{i}) dx^{1} \wedge \dots \wedge dx^{n}
  \end{align*}

  Por lo tanto
  \begin{equation*}
    \dive X = \frac{\sum_{i = 1}^{n} \frac{\partial}{\partial x^{i}} (\sqrt{\det (g_{\ell j})}X^{i})}{\sqrt{\det(g_{ij})}}
  \end{equation*}

  De lo anterior y la formula magica de Cartan tenemos que
  \begin{equation*}
   f \dive X \volg + \inner{\operatorname{grad_{g}}(f)}{X} \volg= f \mathscr{L}_{X}(\volg) + (\mathscr{L}_{f}X)(\volg) = \mathscr{L}_{fX}(\volg) = (d \circ \iota_{fX})(\volg) = \dive (fX) \volg
  \end{equation*}

  Por lo tanto
  \begin{equation*}
    f \dive X + \inner{\operatorname{grad_{g}}(f)}{X} = \dive(fX)
  \end{equation*}

    \item Notemos que
          \begin{equation*}
            \int_{M} \dive X \volg = \int_{M} d(\iota_{X} \volg) = \int_{\partial M} \iota_{X} \volg|_{\partial M}
          \end{equation*}

          Ahora consideremos en cada $p$ una base ortonormal $(e_{2}, \dots, e_{n})$ de $T_{p}\partial M$ tal que $\volg(p) = n \wedge e_{2}^{*} \wedge \dots \wedge e_{n}^{*}$. Calculemos la contracci\'on por $X$ para $v_{1}, \dots v_{n - 1} \in T_{p}\partial M$
          \begin{align*}
            (\hat{n} \wedge e_{2}^{*} \wedge \dots \wedge e_{n}^{*})(a\hat{n} + \sum_{i = 2}^{n} X^{i}e_{i}, v_{1}, \dots, v_{n -1}) &= a(e_{2}^{*} \wedge \dots e_{n}^{*})(v_{1}, \dots, v_{n -1}) + (\hat{n} \wedge e_{2}^{*} \wedge \dots \wedge e_{n}^{*})(\sum_{i = 2}^{n} X^{i}e_{i}, \dots, v_{n - 1})\\
            &= a(e_{2}^{*} \wedge \dots e_{n}^{*})(v_{1}, \dots, v_{n - 1})
          \end{align*}

          Pues en el segundo termino nada tiene direccion normal por lo que al tomar $\hat{n}$ siempre dara 0 sin importar cual vector le demos.

          Por lo tanto teneos que
          \begin{equation*}
            \iota_{X} \volg |_{\partial M} = \inner{X}{\hat{n}} \hat{\volg}
          \end{equation*}

          Con lo que obtenemos
          \begin{equation*}
            \int_{M} \dive X \volg = \int_{\partial M} \inner{X}{\hat{n}} \hat{\volg}
          \end{equation*}

    \item Usando (1), tenemos que
          \begin{equation*}
            \int_{M} \dive(fX) \volg  = \int_{M} f \dive X \volg + \int_{M} \inner{\operatorname{grad_{g}}(f)}{X} \volg
          \end{equation*}

          Por (2) tenemos que
          \begin{equation*}
            \int_{M} \dive(fX) \volg = \int_{\partial M} \inner{fX}{\hat{n}} \hat{\volg} = \int_{\partial M} f \inner{X}{\hat{n}} \hat{\volg}
          \end{equation*}

          Juntando los dos tenemos que
          \begin{equation*}
            \int_{\partial M} f \inner{X}{\hat{n}} \hat{\volg} = \int_{M} f \dive X \volg + \int_{M} \inner{\operatorname{grad_{g}}(f)}{X} \volg
          \end{equation*}

          Reordenando obtenemos que
          \begin{equation*}
            \int_{M} \inner{\operatorname{grad_{g}}(f)}{X} \volg = - \int_{M} f \dive X \volg + \int_{\partial M} f \inner{X}{\hat{n}} \hat{\volg}
          \end{equation*}
  \end{enumerate}
\end{sol}

\end{document}
